\section{SVM}
Support vector machines perform maximum margin linear classification.
Intuitively, this means to maximise the distance between the decision
boundary and both classes.
Formally, the margin is the euclidian distance
between two hyperplanes parallel to the decision boundary.

The points which defined the distance from the decision
boundary are called the support vectors.


\subsection{General SVMs}

SVM uses the Hinge loss which is defined as follows:
\begin{equation*}
    \ell_H(\vec{w}; \vec{x}, y) =
    \max(0, 1 - y \vec{w}^T \vec{x})
\end{equation*}
It only becomes zero if a sample is classified correctly
with a certain "confidence".
The Hinge loss is a convex upper-bound on the
zero-one loss.

The constant $1$ is used in the Hinge loss to ensure
that weights do not explode.
Any other constant can be compensated by simply
scaling the weights accordingly.

The complete SVM objective is
\begin{equation*}
    \hat{\vec{w}} = \arg\min_{\vec{w}} \frac{1}{n}
    \sum_{i=1}^n \max\{0, 1 - y_i \vec{w}^T \vec{x}_i\}
    + \lambda \norm{\vec{w}}_2^2
\end{equation*}
Both the Hinge loss and the regularisation term are required for
the SVM to work.
Furthermore, features have to be standardised.
TODO: Do features really have to be standardised?

The gradient of the Hinge loss is
\begin{equation*}
    \nabla_{\vec{w}} \ell_H(\vec{w}; \vec{x}, y) =
    \begin{cases}
        0 & \text{if $\vec{w}^T \vec{x}_i y_i \geq 1$} \\
        -y_i \vec{x}_i & \text{if $\vec{w}^T \vec{x}_i y_i < 1$}
    \end{cases}
\end{equation*}

Thus, the gradient of the empirical risk for SVMs is
\begin{equation*}
    \nabla_{\vec{w}} \hat{R}(\vec{w})
    = - \frac{1}{n} \sum_{i : \vec{w}^T \vec{x}_i y_i < 1}{
        y_i \vec{x}_i
    }
    + 2 \vec{w}
\end{equation*}

If the data is linearly separable, SVM finds the
maximum margin decision boundary.

A safe learning rate choice is $\eta_t = \frac{1}{\lambda t}$, i.e. use learning rate decay
based on the regularisation parameter $\lambda$.


\subsection{$L_1$-SVM}
In $L_1$-SVM, the penalty term is replaced with a $L_1$ loss,
thus the objective becomes
\begin{equation*}
    \hat{\vec{w}} = \arg\min_{\vec{w}} \frac{1}{n}
    \sum_{i=1}^n \max\{0, 1 - y_i \vec{w}^T \vec{x}_i\}
    + \lambda \norm{\vec{w}}_1
\end{equation*}

The alternative penalty term encourages coefficients to be
exactly zero and thus results in automatic feature selection.


\subsection{Geometric Interpretation}
TODO: Honestly, just read tutorial 4.

A \emph{hyperplane} in $\R^d$ is the
set of vectors fulfilling
$\langle \vec{w}, \vec{x} \rangle = 0$.
It is a $d-1$ dimensional subspace.
For $b \in \R$, the set
$\{\vec{x} \in \R^d \mid \langle \vec{w}, \vec{x} \rangle = b\}$
is a \emph{translated/affine hyperplane}.
Increasing $b$ moves the hyperplane towards
the direction of $\vec{w}$ and vice versa.

The \emph{orthogonal projection} of $\vec{z}$
onto a hyperplane is given by $\hat{\vec{z}} = \vec{z} + t \vec{w}$
where $t = \frac{b - \langle \vec{w}, \vec{z} \rangle}{\norm{\vec{w}}^2}$.

The \emph{orthogonal distance} of $\vec{z}$ to a
hyperplane $H = \{\vec{x} \in \R^d \mid \langle \vec{w}, \vec{x} \rangle = 0\}$
is given by
\begin{equation*}
    \frac{| \langle \vec{w}, \vec{z} \rangle |}{\norm{\vec{w}}^2}
\end{equation*}
If $\vec{w}$ is a unit vector, the distance is directly
given by $|\langle \vec{w}, \vec{z} \rangle|$.

From now on, $\vec{z}$ contains an additional entry
$1$ and $\vec{w}$ contains an additional entry $-b$.
Therefore, all hyperplanes are through the origin.

Let $D$ be a data set and $H$ a hyperplane.
The \emph{margin} of $H$ w.r.t. $D$ is
\begin{equation*}
    \gamma_D(\vec{w}) := \min_{(\vec{x}, y) \in D}{
        \frac{|\langle \vec{w}, \vec{x} \rangle |}{\norm{\vec{w}}}
    }
\end{equation*}

We now assume that
\begin{enumerate}
    \item The data is linearly separable.
    \item There exists a separating hyperplane with non-zero margin.
\end{enumerate}

The \emph{Hard-SVM} problem requires both assumptions.
It is minimising
\begin{equation*}
    \frac{1}{2} \norm{\vec{w}}^2
\end{equation*}
such that
\begin{equation*}
    y_i \cdot \langle \vec{w}, \vec{x}_i \rangle
    \geq 1 \text{ for all $i \in \{1, \dotsc, n\}$}
\end{equation*}

The separating hyperplane assumption can be
dropped by introducing \emph{slack variables}
$\xi_1, \dotsc, \xi_n \geq 0$
which allow to violate some constraints.
$\xi_i$ denotes the amount the
$i$-th constraint is violated.

This leads to the \emph{Soft-SVM} problem.
It is minimising
\begin{equation*}
    \frac{1}{2} \norm{\vec{w}}^2 + C \sum_{i=1}^n{\xi_i}
\end{equation*}
such that
\begin{align*}
    y_i \cdot \langle \vec{w}, \vec{x}_i \rangle
        &\geq 1 - \xi_i & \text{ for all $i \in \{1, \dotsc, n\}$} \\
    \xi_i &\geq 0 & \text{ for all $i \in \{1, \dotsc, n\}$}
\end{align*}
where $C > 0$ denotes the softness of the problem.

If $C = 0$ then $\vec{w} = \vec{0}$ is a trivial
solution; if $C \to \infty$ then the problem
becomes similar to the Hard-SVM problem.

Fixing $\vec{w}$ and optimising $\xi_i$
leads to the solution
\begin{equation*}
    \xi_i^*(\vec{w}) = \ell_H(y_i \cdot \langle \vec{w}, \vec{x}_i \rangle)
\end{equation*}
and by plugging it in to the equivalent
optimisation objective
\begin{equation*}
    \arg\min_{\vec{w}}{
        \frac{1}{2} \norm{\vec{w}}^2
        + C \sum_{i=1}^n{\ell_H(y_i \cdot \langle \vec{w}, \vec{x}_i \rangle)}
    }
\end{equation*}


\subsection{Kernelised SVM}
The reformulisation of the SVM loss (without regulariser)
with respect to inner products and kernels
for sample $i$ is
\begin{equation*}
    \max\{0, 1 - y_i \sum_{j=1}^n{\alpha_j y_j k(\vec{x}_j, \vec{x}_i)}\}
\end{equation*}
and for the regulariser
\begin{equation*}
    \lambda \sum_{i=1}^n\sum_{j=1}^n{
        \alpha_i \alpha_j y_i y_j k(\vec{x}_i, \vec{x}_j)
    }
    = \lambda \vec{\alpha}^T \vec{D}_y \vec{K} \vec{D}_y \vec{\alpha}
\end{equation*}
where $\vec{K}$ is the Gram matrix and
$\vec{D}_y$ contains the $y_i$ on its diagonal.

This leads to the new objective
\begin{equation*}
    \hat{\vec{\alpha}} = \arg\min_{\vec{\alpha}}{
        \frac{1}{n} \sum_{i=1}^n{
            \max\{0, 1 - y_i \vec{\alpha}^T \vec{k}_i\} +
            \lambda \vec{\alpha}^T \vec{D}_y \vec{K} \vec{D}_y \vec{\alpha}
        }
    }
\end{equation*}
where $\vec{k}_i = [y_1 k(\vec{x}_i, \vec{x}_1), \dotsc, y_n k(\vec{x}_i, \vec{x}_n)]$.

Prediction for a data point $\vec{x}$ is done as
\begin{equation*}
    \hat{y} = sign\left(
        \sum_{i=1}^n{\alpha_i y_i k(\vec{x}_i, \vec{x})}
    \right)
\end{equation*}


\subsection{Multi-Class SVM}
SVMs can be generalised to a multi-class scenario.
The key idea is to maintain $c$ weight vectors
$\vec{w}^{(i)}$, one for each class.
The prediction is then
\begin{equation*}
    \hat{y} = \arg\max_i{\vec{w}^{(i)^T} \vec{x}}
\end{equation*}

For each data point, it should hold that the prediction
for the true class is separated by a margin from the
class which has the second highest prediction, i.e.
\begin{equation*}
    \vec{w}^{(y)^T} \vec{x} \geq
    \max_{i \neq y}{\vec{w}^{(i)^T}\vec{x}} + 1
\end{equation*}

The multi-class Hinge loss is given as
\begin{equation*}
    \ell_{MC-H}(\vec{w}^{(1)}, \dotsc, \vec{w}^{(c)}; \vec{x}, y)
    = \max{\{0, 1 +
            \max_{j \neq y}
            {\vec{w}^{(j)^T}\vec{x} - \vec{w}^{(y)^T}\vec{x}}
    \}}
\end{equation*}

The gradient is
\begin{equation*}
    \nabla_{\vec{w}^{(i)}} \ell_{MC-H}(\vec{w}^{(1:c)}; \vec{x}, y) =
    \begin{cases}
        0 & \text{if $\vec{w}^{(y)^T} \vec{x} \geq \max_{i \neq y}{\vec{w}^{(i)^T}\vec{x}} + 1$ or $i \neq y$ and $i \notin \arg\max_j{\vec{w}^{(j)^T} \vec{x}}$} \\
        -x & \text{if $\vec{w}^{(y)^T} \vec{x} < \max_{i \neq y}{\vec{w}^{(i)^T}\vec{x}} + 1$ and $i = y$} \\
        +x & \text{otherwise}
    \end{cases}
\end{equation*}
i.e. zero if either the classification is correct or the
weight does neither correspond to the class nor to the
maximal prediction for the sample;
minus $x$ if the weight corresponds to the class and does
not classify correctly;
plus $x$ if the weight does not correspond to the class
but is the maximum prediction.
