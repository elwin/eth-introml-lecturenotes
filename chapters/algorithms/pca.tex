\section{PCA}
Let
$D = \{\vec{x}_1, \dotsc, \vec{x}_n\}, \vec{x}_i \in \R^d$
be a data set with zero mean,
i.e. $\vec{\mu} = \frac{1}{n} \sum_{i=1}^n{\vec{x}_i} = \vec{0}$.

The \emph{PCA problem} for $k < n$ is finding
\begin{equation*}
    (\vec{W}, \vec{z}_1, \dotsc, \vec{z}_n) = \arg\min{
        \sum_{i=1}^n{\norm{\vec{W} \vec{z}_i - \vec{x}_i}_2^2}
    }
\end{equation*}
where $\vec{W} \in \R^{d \times k}$ is orthogonal and
$\vec{z}_i \in \R^k$.

Let $\vec{\Sigma}$ be the empirical covariance, given as
\begin{equation*}
    \vec{\Sigma} =
    \frac{1}{n} \vec{X}^T \vec{X} =
    \frac{1}{n} \sum_{i=1}^n{
        \vec{x}_i \vec{x}_i^T
    }
\end{equation*}

The solution to the PCA problem in closed-form is given by
\begin{align*}
    \vec{W} &= (\vec{v}_1, \dotsc, \vec{v}_k) \\
    \vec{z}_i &= \vec{W}^T \vec{x}_i
\end{align*}
where $\vec{\Sigma}$ is decomposed as
\begin{equation*}
    \vec{\Sigma} = \vec{V} \vec{D} \vec{V}^T
    = \sum_{i=1}^d{\lambda_i \vec{v}_i \vec{v}_i^T}
\end{equation*}
and the $\lambda_i$ are ordered in descending order,
$\lambda_1 \geq \dotsb \geq \lambda_d \geq 0$.

Then, the linear mapping $f(\vec{x}) = \vec{W}^T \vec{x}$
projects $\vec{x} \in \R^d$ to a $k$-dimensional
subspace $\R^k$.
It minimises the reconstruction error with respect to
the Euclidian norm.

Typically, if the data set is clustered,
PCA preserves those clusters.


\subsection{Derivation for $k = 1$}
Assume $\vec{\mu} = \frac{1}{n} \sum_{i=1}^n{\vec{x}_i} = \vec{0}$.

The goal is to represent data points $\vec{x}_i$ on a line
$\vec{w} \in \R^d$ with coefficients $z_i$.
Thus, we want $z_i \vec{w} \approx \vec{x}_i$.

To get a unique solution, we further require $\vec{w}$
to be normalised, i.e. $\norm{\vec{w}}_2 = 1$.
Then, we can jointly optimise
\begin{equation*}
    (\vec{w}^*, \vec{z}^*) = \arg\min_{\norm{\vec{w}}_2 = 1, \vec{z}}{
        \underbrace{
            \sum_{i=1}^n{\norm{z_i \vec{w} - \vec{x}_i}_2^2}
        }_{\hat{R}(\vec{w}, \vec{z})}
    }
\end{equation*}

For a fixed $\vec{w}$, with $\norm{\vec{w}}_2 = 1$,
the optimal $z_i$ is given as the
orthogonal projection of $\vec{x}_i$ onto $\vec{w}$, i.e.
\begin{equation*}
    z_i^* = \vec{w}^T \vec{x}_i
\end{equation*}
This leads to the equivalent optimisation objective
\begin{equation*}
    \vec{w}^* = \arg\min_{\norm{\vec{w}}_2 = 1}{
        \underbrace{
            \sum_{i=1}^n{\norm{\vec{w} \vec{w}^T \vec{x}_i - \vec{x}_i}_2^2}
        }_{\hat{R}(\vec{w})}
    }
\end{equation*}

Now, the empirical risk can be reformulated:
\begin{align*}
    \hat{R}(\vec{w}) &= \sum_{i=1}^n{
        (\vec{w} \vec{w}^T \vec{x}_i - \vec{x}_i)^T
        (\vec{w} \vec{w}^T \vec{x}_i - \vec{x}_i)
    } \\
    &= \sum_{i=1}^n{
        \vec{x}_i^T \vec{w} \underbrace{\vec{w}^T \vec{w}}_{=1} \vec{w}^T \vec{x}_i
        - 2 \vec{x}_i^T \vec{w} \vec{w}^T \vec{x}_i
        + \vec{x}_i^T \vec{x}_i
    } \\
    &= \sum_{i=1}^n{
        \underbrace{(\vec{w}^T \vec{x}_i)^T (\vec{w}^T \vec{x}_i)}_{= (\vec{w}^T \vec{x}_i)^2}
        - 2 \underbrace{(\vec{w}^T \vec{x}_i)^T (\vec{w}^T \vec{x}_i)}_{= (\vec{w}^T \vec{x}_i)^2}
        + \underbrace{\vec{x}_i^T \vec{x}_i}_{= \norm{\vec{x}_i}_2^2}
    } \\
    &= \underbrace{\sum_{i=1}^n{\norm{\vec{x}_i}_2^2}}_{
        = \text{ const w.r.t. $\vec{w}$}
    }
    - \sum_{i=1}^n{(\vec{w}^T \vec{x}_i)^2}
\end{align*}

Thus, minimising the previous objective is equivalent to
maximising the new objective
\begin{equation*}
    \vec{w}^* = \arg\max_{\norm{\vec{w}}_2 = 1}{
        \sum_{i=1}^n{(\vec{w}^T \vec{x}_i)^2}
    }
\end{equation*}

By reformulating
\begin{equation*}
    \sum_{i=1}^n{(\vec{w}^T \vec{x}_i)^2}
    = \sum_{i=1}^n{\vec{w}^T \vec{x}_i \vec{x}_i^T \vec{w}}
    = \vec{w}^T \underbrace{\sum_{i=1}^n{(\vec{x}_i \vec{x}_i^T)}}_{= n \vec{\Sigma}} \vec{w}
\end{equation*}
this is again equivalent to the final objective
\begin{equation*}
    \vec{w}^* = \arg\max_{\norm{\vec{w}}_2 = 1}{
        \vec{w}^T \vec{\Sigma} \vec{w}
    }
\end{equation*}
where $\vec{\Sigma} = \frac{1}{n} \sum_{i=1}^n{\vec{x}_i \vec{x}_i^T}$ is the empirical covariance assuming the
data is centred.

We can take the eigendecomposition of $\vec{\Sigma}$ and
write it as
$\vec{\Sigma} = \sum_{i=1}^d{\lambda_i \vec{v}_i \vec{v}_i^T}$
with $\lambda_1 \geq \dotsb \geq \lambda_d \geq 0$.
$\vec{w}$ can also be represented in the eigenbasis
of $\vec{\Sigma}$ as
$\vec{w} = \sum_{i=1}^d{\alpha_i \vec{v}_i}$.
All $\alpha_i \geq 0$ and must sum to $1$,
are thus a probability distribution.

Now, the objective can be rewritten as
\begin{align*}
    \vec{w}^T \vec{\Sigma} \vec{w}
    &= \left(
        \sum_{i=1}^d{\alpha_i \vec{v}_i}
    \right)^T \left(
        \sum_{j=1}^d{\lambda_j \vec{v}_j \vec{v}_j^T}
    \right) \left(
        \sum_{k=1}^d{\alpha_k \vec{v}_k}
    \right) \\
    &= \sum_{i=1}^d{\sum_{j=1}^d{\sum_{k=1}^d{
        \alpha_i \alpha_k \lambda_j
        \underbrace{(\vec{v}_i^T \vec{v}_j)}_{\delta_{i,j}}
        \underbrace{(\vec{v}_j^T \vec{v}_k)}_{\delta_{j,k}}
    }}} \\
    &= \sum_{i=1}^d{\alpha_i^2 \lambda_i}
\end{align*}

From the constraint $\norm{\vec{w}}_2 = 1$ we have
$\sum_{i=1}^d{\alpha_i^2} = 1$,
thus the optimal solution is given by
$\alpha_1 = 1$ and $\alpha_i = 0$ for all $i > 1$
because $\lambda_1$ is the maximum value.

Therefore, the maximum is given by the principal eigenvector
of $\vec{\Sigma}$.
This generalises for $k > 1$ principal components.


\subsection{Connection to SVD}
Let $\vec{X} \in \R^{n \times d}$ be the data matrix,
centred.
Furhtermore, let $\vec{X} = \vec{U} \vec{S} \vec{V}^T$ be
the SVD of $\vec{X}$ where $\vec{U} \in \R^{n \times n}$
and $\vec{V} \in \R^{d \times d}$ are orthogonal,
and $\vec{S} \in \R^{n \times d}$ contains the singular
values in decreasing order on its diagonal.

The top $k$ principal components are then the first $k$
columns of $\vec{V}$:
\begin{equation*}
    n \vec{\Sigma}
    = \vec{X}^T \vec{X}
    = \vec{V} \vec{S}^T \underbrace{\vec{U}^T \vec{U}}_{=\vec{I}} \vec{S} \vec{V}^T
    = \vec{V} \underbrace{\vec{S}^T \vec{S}}_{= \vec{D}} \vec{V}^T
\end{equation*}


\subsection{Model Selection}
Cross-validation can be used to select the parameter $k$
by evaluating the cost.

Another approach is to pick $k$ so that most of the variance
is explained.
TODO: How?


\subsection{Comparison to k-means}
There exists a equivalent formulation for the k-means problem
which is very similar to the PCA problem.

The PCA problem is
\begin{equation*}
    (\vec{W}, \vec{z}_1, \dotsc, \vec{z}_n) = \arg\min{
        \sum_{i=1}^n{\norm{\vec{W} \vec{z}_i - \vec{x}_i}_2^2}
    }
\end{equation*}
where $\vec{W} \in \R^{d \times k}$ is orthogonal and
$\vec{z}_i \in \R^k$.

The k-means problem is
\begin{equation*}
    (\vec{W}, \vec{z}_1, \dotsc, \vec{z}_n) = \arg\min{
        \sum_{i=1}^n{\norm{\vec{W} \vec{z}_i - \vec{x}_i}_2^2}
    }
\end{equation*}
where $\vec{W} \in \R^{d \times k}$ is arbitrary and
$\vec{z}_i \in E_k$ where $E_k$ is the set of unit vectors
in $\R^k$.

The discrete nature of the k-means problem makes it NP-hard.
Conversely, PCA can be used to initialise k-means cluster centres.

The line between the centroids for $k=2$ approximates
the first principal component,
the plane between the centroids for $k=3$ the first
two principal components, etc.


\subsection{Kernelised PCA}
For non-linear problems, PCA can be kernelised.

Let $k \geq 1$ be the target dimensionality
and $\vec{K} \in \R^{n \times n}$
be the kernel/Gram matrix and its eigendecomposition
$\vec{K} = \sum_{i=1}^n{\lambda_i \vec{v}_i \vec{v}_i^T}$
with $\lambda_1 \geq \dotsb \geq \lambda_n \geq 0$.

The \emph{kernel principal components} are given by
$\vec{\alpha}^{(1)}, \dotsc, \vec{\alpha}^{(k)} \in \R^n$
where
\begin{equation*}
    \vec{\alpha}^{(i)} = \frac{1}{\sqrt{\lambda_i}}
    \vec{v}_i
\end{equation*}

Given this, a new point $\vec{x}$ is projected as
$\vec{z} \in \R^k$ with
\begin{equation*}
    z_i = \sum_{j=1}^n{
        \alpha_j^{(i)} k(\vec{x}, \vec{x}_j)
    }
\end{equation*}

The kernel has to be centred where
$\vec{E} = \frac{1}{n} [1, \dotsc, 1] [1, \dotsc, 1]^T$ as
\begin{equation*}
    \vec{K}' = \vec{K} - \vec{K}\vec{E} - \vec{E}\vec{K} + \vec{E}\vec{K}\vec{E}
\end{equation*}

\emph{Kernel-k-means} refers to applying k-means onto
kernel principal components.


\subsubsection{Derivation for $k=1$}
For PCA, $\vec{w}^*$ is the leading eigenvector $\lambda$
of $\vec{X}^T \vec{X}$.
Thus,
\begin{equation*}
    \vec{X}^T \underbrace{\vec{X} \vec{w}}_{\vec{\beta}}
    = \lambda \vec{w}
    \Rightarrow
    \vec{w} = \frac{1}{\lambda} \vec{X}^T \vec{\beta}
        = \frac{1}{\lambda} \sum_{i=1}^{n}{\beta_i \vec{x}_i}
\end{equation*}

Therefore, with $\alpha_i = \frac{\beta_i}{\lambda}$,
the general Ansatz holds:
\begin{equation*}
    \vec{w} = \sum_{i=1}^n{\alpha_i \vec{x}_i}
\end{equation*}
Now, both the objective and constraint have to be expressed in
terms of $\alpha$ and inner products.

Let $\vec{K}$ be the kernel/Gram matrix
($\vec{K}_{i,j} = k(\vec{x}_i, \vec{x}_j)$)
and $\vec{K}_i$ its $i$-th column.

For the objective, we have
\begin{align*}
    \vec{w}^T \vec{X}^T \vec{X} \vec{w} &=
        \sum_{i=1}^n{(\vec{w}^T \vec{x}_i)^2} \\
    &= \sum_{i=1}^n{\left((
        \sum_{j=1}^n{\alpha_j \vec{x}_j}
    )^T \vec{x}_i\right)^2} \\
    &= \sum_{i=1}^n{\left(
        \sum_{j=1}^n{\alpha_j \vec{x}_j^T \vec{x}_i}
    \right)^2} \\
    &= \sum_{i=1}^n{\left(
        \sum_{j=1}^n{\alpha_j k(\vec{x}_j, \vec{x}_i)}
    \right)^2} \\
    &= \sum_{i=1}^n{\left(
        \vec{\alpha}^T \vec{K_i}
    \right)^2} \\
    &= \vec{\alpha}^T \vec{K}^T \vec{K} \vec{\alpha}
\end{align*}

For the constraints, we have
\begin{align*}
    \norm{\vec{w}}_2 &= \vec{w}^T \vec{w} \\
    &= \left(\sum_{i=1}^n{\alpha_i \vec{x}_i}\right)^T
        \left(\sum_{i=1}^n{\alpha_i \vec{x}_i}\right) \\
    &= \sum_{i=1}^n{
        \sum_{j=1}^n{
            \alpha_i \alpha_j \vec{x}_i^T \vec{x}_j
        }
    } \\
    &= \sum_{i=1}^n{
        \sum_{j=1}^n{
            \alpha_i \alpha_j k(\vec{x}_i, \vec{x}_j)
        }
    } \\
    &= \sum_{i=1}^n{
        \alpha_i \vec{K}_i \vec{\alpha}
    } \\
    &= \vec{\alpha}^T \vec{K} \vec{\alpha} \\
    &\overset{!}{=} 1
\end{align*}

Thus, the goal is
\begin{equation*}
    \vec{\alpha}^* = 
    \arg\max_{\vec{\alpha}^T \vec{K} \vec{\alpha}}{
        \vec{\alpha}^T \vec{K}^T \vec{K} \vec{\alpha}
    }
\end{equation*}

The optimal solution is then obtained by
\begin{equation*}
    \vec{\alpha}^* =
    \frac{1}{\sqrt{\lambda_1}} \vec{v}_1
\end{equation*}

Scaling is required because
\begin{align*}
    \vec{\alpha}^{*^T} \vec{K} \vec{\alpha}^* &=
        \frac{1}{\sqrt{\lambda_1}} \vec{v}_1^T
        \vec{K}
        \frac{1}{\sqrt{\lambda_1}} \vec{v}_1 \\
    &= \frac{1}{\lambda_1} \vec{v}_1^T \vec{K} \vec{v}_1 \\
    &= \frac{1}{\lambda_1} \vec{v}_1^T (\lambda_1 \vec{v}_1) \\
    &= \vec{v}_1^T \vec{v}_1 \\
    &= 1
\end{align*}


\subsection{Maximum Variance Perspective}
Instead of minimising the reconstruction error,
PCA can also be interpreted as maximising the
variance of the projected data.

The \emph{projected variance} is
\begin{equation*}
    Q_R(\vec{\Sigma}, \vec{w}) =
    \frac{1}{n} \sum_{i=1}^n{\norm{
        \vec{w}^T \vec{x}_i - \vec{w}^T \bar{\vec{x}}
    }_2^2}
    = \vec{w}^T \vec{\Sigma} \vec{w}
\end{equation*}
where $\bar{\vec{x}} = \frac{1}{n} \sum_{i=1}^n{\vec{x}_i}$
is the data mean and
$\vec{\Sigma} = \frac{1}{n}\sum_{i=1}^n{(\vec{x}_i - \bar{\vec{x}}) (\vec{x}_i - \bar{\vec{x}})^T}$
the data covariance.
If $\bar{\vec{x}} = \vec{0}$ then
$\vec{\Sigma} = \frac{1}{n} \vec{X}^T \vec{X}$.

Now, if we want to maximise the projected variance,
\begin{equation*}
    \vec{w}^* = \arg\max_{\vec{w}}{
        \vec{w}^T \vec{\Sigma} \vec{w} + \lambda (1 - \vec{w}^T \vec{w})
    }
\end{equation*}
by setting the derivative to zero we get
\begin{align*}
    \frac{\partial}{\partial \vec{w}}
    \vec{w}^T \vec{\Sigma} \vec{w} + \lambda (1 - \vec{w}^T \vec{w}) &= 2 \vec{\Sigma} \vec{w} - 2 \lambda \vec{w} = 0 \\
    \Leftrightarrow \vec{\Sigma} \vec{w}
    &= \lambda \vec{w} \\
    \Leftrightarrow \vec{w}^T \vec{\Sigma} \vec{w} &= \lambda
\end{align*}
Thus, $\vec{w}$ is an eigenvector of $\vec{\Sigma}$ and
$\vec{w}^T \vec{\Sigma} \vec{w} = \lambda$,
i.e. the variance is given by the eigenvalue of $\vec{\Sigma}$.

The PCA solution selects the eigenvector of $\vec{\Sigma}$
with the largest eigenvalue $\lambda$.
Thus, the PCA solution maximises the projected variance.
The derivation in the $k$-dimensional case is equivalent,
except that the trace of the arg max is taken.

It holds that
variance plus error squared is constant.
Thus, maximising the variance corresponds to minimising
the error.
