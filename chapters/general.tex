\chapter{General}

\section{Terminology}
The \emph{loss} $\ell(f(\vec{x}_i), \vec{y}_i)$ is defined
on a single data point,
the \emph{cost} on the whole data set.
The cost might also include penalty terms.

The \emph{objective} refers to any function
optimised during training.

\emph{Regression} has the goal to predict real-valued
labels/vectors, in general a mapping
$f : \mathbb{R}^d \to \mathbb{R}$.

In \emph{classification}, the target variable
$Y$ is discrete, i.e. categorical.

In \emph{supervised learning}, data is available
as input-output pairs.

A \emph{decision rule}/\emph{hypothesis} is a rule
to assign a class to a sample.

The \emph{hypothesis / hypothesis class} is the
function / class of functions used to fit data.

A \emph{surrogate loss} is used whenever we want
to optimise a loss which is intractable.
The surrogate loss approximates the desired
loss during optimisation.
The target loss is used during evaluation.

\emph{Parametric} models have a constant number of parameters.
Nonparametric models grow in complexity
with the size of data.
Nonparametric models are potentially much more expressive
but also more computationally complex.


TODO: Difference between derivative and gradient,
especially column/row vector.


\section{Gradient Descent}
Let $\vec{w}_0 \in \R^d$ be an arbitrary
starting point,
$\eta_t$ the learning rate at time $t$ and
$\hat{R}(\vec{w})$ be a differentiable function
which should be minimised.
For $t = 1, 2, \dotsc$ gradient descent calculates
\begin{equation*}
    \vec{w}_{t + 1} = \vec{w}_t - \eta_t
    \nabla \hat{R}(\vec{w}_t)
\end{equation*}

Under mild assumptions, if the step size is sufficiently
small, gradient descent converges to a stationary
point.
Thus, for convex objectives, it finds the optimal solution (if one exists).

If the step size is chosen too small, convergence may
take very long.
If the step size is too large, the objective might
oscillate or even diverge.
The step size can also be chosen adaptively.

The line search heuristic works as follows:
Let $g_t = \nabla \hat{R}(\vec{w}_t)$.
Pick $\eta_t \in \arg\min_{\eta \in (0, \infty)}(
    \hat{R}(\vec{w}_t - \eta \cdot g_t)
)$,
i.e. the step size which minimises the objective
w.r.t. the current gradient.

The bold driver heuristic works as follows:
\begin{itemize}
    \item If the function decreases, increase the step size:
    \begin{equation*}
        \hat{R}(\underbrace{\vec{w}_t - \eta_t g_t}_{\vec{w}_{t+1}}) < \hat{R}(\vec{w}_t)
        \Rightarrow \eta_{t + 1} = \eta_t \cdot c_{acc}
    \end{equation*}
    \item If the function increases, decrease the step size:
    \begin{equation*}
        \hat{R}(\vec{w}_t - \eta_t g_t) > \hat{R}(\vec{w}_t)
        \Rightarrow \eta_{t + 1} = \eta_t \cdot c_{dec}
    \end{equation*}
\end{itemize}

Stochastic gradient descent uses only a single
sample per iteration.
The sample is picked uniform at random
(with replacement) from the dataset,
used to compute the gradient and to update the
weights.

Stochastic gradient descent is guaranteed to
converge under mild assumptions if
$\sum_t{\eta_t} = \infty$ and
$\sum_t{\eta_t^2} < \infty$.

Using only a single sample at a time is computationally inefficient
and might have large variance in the gradient estimate.
It might thus lead to slow convergence.

Using mini-batch SGD reduces variance.
It uses mini-batches and averages the gradients with respect to
multiple randomly selected points.

\section{Model Selection}
Underfitting happens if the model is too
simple, overfitting if the model is too complex.

As the model complexity increases,
the training error usually decreases.
However, the prediction error usually
decreases up to a point, and then starts
increasing again.


\subsection{Risks}
We usually assume that the data set is generated
independently and identically distributed (i.i.d.)
from some unknown distribution $P$,
$(\vec{x}_i, y_i) \sim P(\vec{X}, Y)$.
The i.i.d. assumption can be invalid, e.g. if
\begin{itemize}
    \item Time series data
    \item Spatially correlated data
    \item Correlated noise
\end{itemize}
The goal is then to minimise the expected error
(true risk) under $P$.
The empirical risk is used to estimate the true risk.

In the following, $\ell(y, \hat{y})$ is a loss function
and $\hat{y} = f(\vec{x}; \theta)$ is a prediction.

The population risk/true risk/expected error under $P$ is
\begin{equation*}
    R(\vec{w})
    = \int{P(\vec{x}, y)\ell(y, \hat{y}) d\vec{x} dy}
    = \Exp_{\vec{x}, y}[\ell(y, \hat{y})]
\end{equation*}

The true risk on a sample data set $D$ is
\begin{equation*}
    \hat{R}_D(\vec{w}) = \frac{1}{|D|}
        \sum_{(\vec{x}, y) \in D}{\ell(y, \hat{y})}
\end{equation*}

Under the i.i.d. assumption,
according to the law of large numbers,
$\hat{R}_D(\vec{w}) \to R(\vec{w})$
for any fixed $\vec{w}$ as $|D| \to \infty$.

Empirical risk minimisation on training data $D$ is
finding
$\hat{\vec{w}}_D = \arg\min_{\vec{w}}{\hat{R}_D(\vec{w})}$.
Ideally, we want to solve
$\vec{w}^* = \arg\min_{\vec{w}}{R(\vec{w})}$.
Generalisation refers to a model which extracts
all relevant information from the training data
but also performs well on unseen data.
TODO: What is the generalisation error? Is it $R(\hat{\vec{w}}_D)$?

For learning via empirical risk minimisation,
uniform convergence is needed:
\begin{equation*}
    \sup_{\vec{w}}{|R(\vec{w}) - \hat{R}(\vec{w})|} \to 0
    \text{ as $|D| \to \infty$}
\end{equation*}
This is not implied by the law of large numbers
and depends on the model class.
For example, if "bad" points converge slower than
"good" points, the generalisation error might increase
with increasing amounts of data.

In general, it holds that
\begin{equation*}
    \Exp_D[\hat{R}_D(\hat{\vec{w}}_D)] \leq
    \Exp_D[R(\hat{\vec{w}}_D)]
\end{equation*}
Thus, when evaluating performance on the training data,
an overly optimistic estimate is achieved.

The general idea is thus to use two independent data sets
$D_{train} \sim P$ and $D_{test} \sim P$.
Optimisation happens on the training set, i.e.
\begin{equation*}
    \hat{\vec{w}} = \arg\min_{\vec{w}}{\hat{R}_{train}(\vec{w})}
\end{equation*}
and evaluation on the test set
\begin{equation*}
    \hat{R}_{test}(\hat{\vec{w}}) = \frac{1}{|D_{test}|}
    \sum_{(\vec{x}, y) \in D_{test}}{
        (y - \hat{\vec{w}}^T \vec{x})^2
    }
\end{equation*}
Then,
\begin{equation*}
    \Exp_{D_{train}, D_{test}}[\hat{R}_{test}(\hat{\vec{w}}_{D_{train}})] =
    \Exp_{D_{train}}[R(\hat{\vec{w}}_{D_{train}})]
\end{equation*}

The test error itself is random.
The variance increases for more complex models.
Thus, using a single test set creates bias.


\subsection{Cross-Validation}
Using multiple test sets wastes data.
Thus, for each candidate model $m$
and $i = 1, \dotsc, k$:
\begin{enumerate}
    \item Split $D = D_{train}^{(i)} \uplus D_{val}^{(i)}$
    \item Train model:
    $\hat{\vec{w}}_{i,m} = \arg\min_{\vec{w}}{\hat{R}_{train}^{(i)}(\vec{w})}$
    \item Estimate error:
    $\hat{R}_{m}^{(i)} = \hat{R}_{val}^{(i)}(\hat{\vec{w}}_{i, m})$
\end{enumerate}
The model can then be selected as
\begin{equation*}
    \hat{m} = \arg\min_m{\frac{1}{k}
        \sum_{i=1}^k{\hat{R}_m^{(i)}}
    }
\end{equation*}

Monte Carlo cross-validation picks a training set of
given size uniformly at random and validates on the
remaining points.
The prediction error is then estimated over multiple
random trials.

$k$-fold cross-validation partitions the training set
into $k$ folds.
$k - 1$ are used for training, $1$ for validation.
The prediction error is then estimated as the mean
validation error while varying the validation folds.
$k = 1$ is called leave-one-out cross-validation (LOOCV).

For large enough $k$, the cross-validation error estimate
is very nearly unbiased.

If $k$ is picked too small, there is little data for
training and a risk to underfit the training set and
overfit the test set.
If $k$ is picked too large, performance is usually better
but the computational complexity increases.
In practice, $k = 5$ or $k = 10$ often works well.

However, the error on the validation set will usually
also be underestimated.
If a model is selected, it is usually retrained with
the full data set.

\section{Standardisation}
Standardisation ensures that each feature has
zero mean and unit variance.

Let $\Tilde{x}_{i,j}$ be the $j$-th feature of
the $i$-th data point. Then
\begin{align*}
    \Tilde{x}_{i,j} &= \frac{x_{i,j} - \hat{\mu}_j}{\hat{\sigma}_j} \\
    \hat{\mu}_j &= \frac{1}{n} \sum_{i=1}^n{x_{i,j}} \\
    \hat{\sigma}_j^2 &= \frac{1}{n} \sum_{i=1}^n{(x_{i,j} - \hat{\mu}_j)^2}
\end{align*}

\section{Feature Selection}
There are various reasons to
not use all available features, e.g. because of
\begin{itemize}
    \item Interpretability
    \item Generalisation
    \item Storage / computation / cost
\end{itemize}

A naïve approach would be to try out all subsets of
features.
However, by using greedy feature selection,
computational cost is massively reduced.

In the following,
let $V = \{1, \dotsc, d\}$ be the set of all features
and $\hat{L}(S)$ be the cross-validation error
using features in $S \subseteq V$ only.


\subsection{Greedy Forward Selection}
Start with $S = \emptyset$ and $E_0 = \infty$.

Then, for $i = 1, \dotsc, d$,
find the best element to add
\begin{equation*}
    s_i = \arg\min_{j \in V \setminus S}{
        \hat{L}(S \cup \{j\})
    }
\end{equation*}
and compute the new error
\begin{equation*}
    E_i = \hat{L}(S \cup \{s_i\})
\end{equation*}
If the new error increases, $E_i > E_{i-1}$,
break, else update $S \gets S \cup \{s_i\}$.

The problem with greedy forward selection is if there is
some combination of features which improve the model together
but do not improve it (or even make it worse) if used alone,
i.e. dependent features.


\subsection{Greedy Backward Selection}
Start with $S = V$ and $E_{d + 1} = \infty$.

Then, for $i = d, \dotsc, 1$,
find the best element to remove
\begin{equation*}
    s_i = \arg\min_{j \in S}{
        \hat{L}(S \setminus \{j\})
    }
\end{equation*}
and compute the new error
\begin{equation*}
    E_i = \hat{L}(S \setminus \{s_i\})
\end{equation*}
If the new error increases, $E_i > E_{i-1}$,
break, else update $S \gets S \setminus \{s_i\}$.


\subsection{Sparsity Trick in Linear Models}
Explicitly selecting $k$ features is equivalent to constraining
$\vec{w}$ to have at most $k$ non-zero entries in linear models,
i.e. to be sparse.

The $\norm{\vec{w}}_0$ is defined as the number of non-zero
entries in $\vec{w}$.
Then, we want to optimise the objective function so that
$\norm{\vec{w}}_0 \leq k$.
Alternatively, a penalty term $\lambda \norm{\vec{w}}_0$
can be added to the objective.
However, this is a hard combinatorial optimisation problem.

The $L_1$ loss can act as a surrogate for the $L_0$ loss.
This is called the sparsity trick.

\emph{Fisher consistency} is defined as follows:
Let $\psi : \mathcal{Y} \times \mathcal{S} \to \mathbb{R}$
be a surrogate loss and
$\ell : \mathcal{Y} \times \mathcal{S} \to \mathbb{R}$
be a loss.
$\psi$ is consistent with respect to $\ell$
if every minimiser $f$ of the surrogate risk
$R_\psi(f)$ is also a minimiser of the
risk function $R_\ell(f)$.

The $L_1$ norm is convex and thus,
combined with convex losses,
results in a convex optimisation problem.

While SGD can be used in principle,
convergence is usually slow and coefficients
are rarely exactly zero.

During cross-validation, a good approach is to start with
a large $\lambda$ (s.t. pretty much all weights are zero)
and then gradually decrease it.


\subsection{Comparison}
For greedy methods,
forward selection is usually faster if there are few relevant
features, but it can miss dependencies.
Backward selection can handle dependent features but may be
very slow.
Both methods are only heuristics and can result in non-optimal
selections.
Also, both methods can result in high computational cost.

$L_1$-regularisation is faster because training and feature
selection happen jointly,
but it only works for linear models.

\section{Kernels}
To find non-linear classification boundaries in linear
classifiers, non-linear transformations of the feature vectors
followed by linear classification can be used.
In the case of, for example, polynomials however,
$\mathcal{O}(d^k)$ dimensions are required to represent
polynomials of degree $k$ on $d$ features.
Kernels allow to implicitly use such a representation
without the computational cost by implicitly calculating
inner products in high-dimensional spaces.

The fundamental insight is that the optimal hyperplane
lives in the span of the data, i.e.
\begin{equation*}
    \hat{\vec{w}} = \sum_{i=1}^n{\alpha_i y_i \vec{x}_i}
\end{equation*}
This follows from the Representer Theorem.
Handwavily, starting gradient descent from $\vec{0}$
constructs and maintains such a representation.

The general Ansatz is to write
\begin{equation*}
    \vec{w} = \sum_{i=1}^n{y_i \alpha_i \vec{x}_i}
\end{equation*}
and recalculating the optimisation problems.
From that follows that the objective only depends
on inner products $\vec{x}_i^T \vec{x}_j$.

A kernel $k$ is a function which computes
\begin{equation*}
    k(\vec{x}, \vec{x}') :=
    \phi(\vec{x})^T \phi(\vec{x}')
\end{equation*}
Usually, $k$ can be computed much more efficiently than
$\phi(\vec{x})^T \phi(\vec{x'})$.

The \emph{kernel trick} is expressing a problem such that
it depends only on inner products and then replacing the
inner products by kernels:
\begin{equation*}
    \vec{x}_i^T \vec{x}_j \Rightarrow k(\vec{x}_i, \vec{x}_j)
\end{equation*}


\subsection{Properties of Kernels}
Let $\mathcal{X}$ be a data space.

A kernel is a function
$k : \mathcal{X} \times \mathcal{X} \to \R$
corresponding to some inner product
$k(\vec{x}, \vec{x}') = \langle \phi(\vec{x}), \phi(\vec{x}') \rangle$
in some suitable space,
satisfying
\begin{enumerate}
    \item Symmetry:
    $\forall \vec{x}, \vec{x}': k(\vec{x}, \vec{x}') = k(\vec{x}', \vec{x})$
    \item Positive semi-definiteness:
    For any $n$ and any set $S = \{\vec{x}_1, \dotsc, \vec{x}_n\} \subseteq \mathcal{X}$,
    the kernel (Gram) matrix $\vec{K}$ must be positive
    semi-definite.
\end{enumerate}

Only positive semi-definiteness (instead of positive definiteness)
is required due to the fact that the concatenation of
functions $\langle \cdot, \cdot \rangle \circ \phi(\cdot)$
is calculated and $\phi$ could be non-injective,
e.g. $\phi(\vec{x}) = 0$.

Alternatively, a symmetric function $k : \mathcal{X} \to \mathcal{X}$
is positive semi-definite if
$\forall n > 0$,
$\forall (a_1, \dotsc, a_n) \in \R^n$,
$\forall (x_1, \dotsc, x_n) \in \mathcal{X}^n$,
\begin{equation*}
    \sum_{i=1}^n{
        \sum_{j=1}^n{
            a_i a_j k(x_i, x_j)
        }
    }
    \geq 0
\end{equation*}

A symmetric matrix $\vec{M} \in \R^{n \times n}$
is positive semi-definite
\begin{equation*}
    \Leftrightarrow \forall \vec{x} \in \R^n:
        \vec{x}^T \vec{M} \vec{x} \geq 0
    \Leftrightarrow
        \text{All Eigenvalues of $\vec{M}$ are} \geq 0
\end{equation*}

Let $k : \mathcal{X} \times \mathcal{X} \to \R$
be a kernel and $S = \{\vec{x}_1, \dotsc, \vec{x}_n\} \subseteq \mathcal{X}$
be a finite data subset.
Then, the \emph{kernel (Gram) matrix} is
\begin{equation*}
    \vec{K} =
    \begin{pmatrix}
    k(\vec{x}_1, \vec{x}_1) & \dots & k(\vec{x}_1, \vec{x}_n) \\
    \vdots & & \vdots \\
    k(\vec{x}_n, \vec{x}_1) & \dots & k(\vec{x}_n, \vec{x}_n)
    \end{pmatrix}
    = \vec{\phi}^T \vec{\phi}
\end{equation*}
where
\begin{equation*}
    \vec{\phi} =
    \left(
    \begin{array}{c|c|c}
    \phi(\vec{x}_1) & \dots & \phi(\vec{x}_n)
    \end{array}
    \right)
\end{equation*}

The kernel (gram) matrix is positive semi-definite.

Conversely, from every symmetric positive semi-definite
matrix $\vec{K} \in \R^{n \times n}$,
we can construct a feature map
$\phi : \mathcal{X} \to \R^n$ such that
$\vec{K}_{i,j} = \phi(i)^T \phi(j)$:
\begin{equation*}
    \vec{K} = \vec{U}\vec{D}\vec{U}^T
    = \underbrace{\vec{U} \vec{D}^{\frac{1}{2}}}_{\vec{\phi}^T}
    \underbrace{\vec{D}^{\frac{1}{2}^T} \vec{U}^T}_{\vec{\phi}}
    =
    \left(
    \begin{array}{c|c|c}
    \phi(1) & \dots & \phi(n)
    \end{array}
    \right)^T
    \left(
    \begin{array}{c|c|c}
    \phi(1) & \dots & \phi(n)
    \end{array}
    \right)
\end{equation*}

\begin{theorem}[Mercer's Theorem]
    Let $\mathcal{X}$ be a compact subset of
    $\R^d$ and
    $k : \mathcal{X} \times \mathcal{X} \to \R$
    a kernel function.
    
    Then, $k$ can be expanded in a uniformly convergent
    series of bounded functions $\phi_i$ so that
    \begin{equation*}
        k(\vec{x}, \vec{x}') =
        \sum_{i=1}^{\infty}{
            \lambda_i \phi_i(\vec{x}) \phi_i(\vec{x}')
        }
    \end{equation*}
\end{theorem}


\subsection{Common Kernels}
The Gaussian and Laplacian kernels compute an inner product
in an infinite data space.
In the following, $\mathcal{X}$ is the data space,
$\mathcal{H}$ the inner product space
(with $\langle \cdot , \cdot \rangle_\mathcal{H}$)
and $\phi(\cdot)$ the feature map.

\subsubsection{Linear Kernel}
\begin{equation*}
    k(\vec{x}, \vec{x}') = \vec{x}^T \vec{x}'
\end{equation*}
$\mathcal{X} = \R^n$,
$\mathcal{H} = \R^n$,
$\phi(\vec{x}) = \vec{x}$.

\subsubsection{Monomials of Degree $m$}
\begin{equation*}
    k(\vec{x}, \vec{x}') = (\vec{x}^T \vec{x}')^m
\end{equation*}

Direct calculation is of order $\mathcal{O}(d^m)$
while the kernel is of order $\mathcal{O}(d)$.

\subsubsection{Monomials of Degree up to $m$}
\begin{equation*}
    k(\vec{x}, \vec{x}') = (1 + \vec{x}^T \vec{x}')^m
\end{equation*}
$\mathcal{X} = \R^n$,
$\mathcal{H} = \R^{\binom{n+m}{m}}$.

The number of monomials up to degree $m$ is
$\mathcal{O}(\binom{d+m}{m})$.

\subsubsection{Gaussian/RBF/Squared Exponential Kernel}
\begin{equation*}
    k(\vec{x}, \vec{x}') = \exp\left(
        -\frac{\norm{\vec{x} - \vec{x}'}_2^2}{h^2}
    \right)
\end{equation*}
$\mathcal{X} = \R^n$,
$\mathcal{H} = \R^\infty$.

$h$ is the bandwitdh/lengthscale parameter.
The smaller $h$, the more influence do samples have.
Bigger $h$ lead to smoother functions.
If the bandwith is increased,
it starts to behave like a linear kernel.

\subsubsection{Laplacian Kernel}
\begin{equation*}
    k(\vec{x}, \vec{x}') = \exp\left(
        -\frac{\norm{\vec{x} - \vec{x}'}_1}{h}
    \right)
\end{equation*}
$h$ is the bandwitdh/lengthscale parameter.
The smaller $h$, the more influence do samples have.
Bigger $h$ lead to smoother functions.
The Laplacian kernel decision boundary behaves
sort of piecewise linear.


\subsubsection{Matrix Kernel}
\begin{equation*}
    k(\vec{x}, \vec{x}') = \vec{x}^T \vec{M} \vec{x}
\end{equation*}
is a kernel if and only if $\vec{M}$ is
symmetric, positive semi-definite.

\subsubsection{ANOVA Kernel}
\begin{equation*}
    k(\vec{x}, \vec{x}') = 
    \sum_{j=1}^d{
        k_j(x_j, x_j')
    }
\end{equation*}
where $x \in \R^d$,
$k : \R^d \times \R^d \to \R$ and
$k_j : \R \times \R \to \R$
are valid kernels.

\subsubsection{Semi-Parametric Regression Kernel}
Often, parametric models are too rigid and non-parametric
models fail to interpolate.
Thus, an additive combination of linear and non-linear
kernels can be used:
\begin{equation*}
    k(\vec{x}, \vec{x}') =
    c_1 \exp\left(
            -\frac{\norm{\vec{x} - \vec{x}'}_2^2}{h^2}
        \right) +
    c_2 \vec{x}^T \vec{x}'
\end{equation*}


\subsection{Kernel Engineering}
Let $k_1 : \mathcal{X} \times \mathcal{X} \to \R$
and $k_2 : \mathcal{X} \times \mathcal{X} \to \R$
be two kernels,
$c > 0$ be a scalar,
$f : \R \to \R$
be either a polynomial with positive coefficients or the
exponential function and
$\mathcal{V} : \mathcal{Z} \to \mathcal{X}$ be a mapping.

Then, the following are also valid kernels:
\begin{itemize}
    \item $k(\vec{x}, \vec{x}') = k_1(\vec{x}, \vec{x}') + k_2(\vec{x}, \vec{x}')$
    \item $k(\vec{x}, \vec{x}') = k_1(\vec{x}, \vec{x}') \cdot k_2(\vec{x}, \vec{x}')$
    \item $k(\vec{x}, \vec{x}') = c \cdot k_1(\vec{x}, \vec{x}')$
    \item $k(\vec{x}, \vec{x}') = f(k_1(\vec{x}, \vec{x}'))$
    \item $k(\vec{z}, \vec{z}') = k_1(\mathcal{V}(\vec{z}), \mathcal{V}(\vec{z}'))$
\end{itemize}


\subsection{Interpretation}
Kernels can be interpreted as similarity functions.
They compute a function around each data point.
The resulting function is then a linear combination of
those functions, scaled by the $\alpha_i$s.

\section{Classification Metrics and Class Imbalance}

\subsection{Class Imbalance}
Generally, the positive class is assumed to be rare.
There are multiple ways class imbalance can be mitigated:

\begin{description}
    \item[Subsampling] is removing majority class samples
    from the data set (e.g. uniformly at random).
    \item[Upsampling] is repeating data points from the
    minority class, possibly with small random perturbations.
    \item[Cost-sensitive classification] methods change
    the loss function to consider the class distribution.
\end{description}

Downsampling results in a smaller data set, thus training
becomes faster. However, data is wasted and information about
the majority class may be lost.
Upsampling makes use of all data but is slower and
perturbations requires arbitrary choices.
Cost-sensitive classification methods are a solution
to those problems.

In cost-sensitive classification,
the loss $\ell(\vec{w}; \vec{x}, y)$ is replaced as
\begin{equation*}
    \ell_{CS} = c_y \ell(\vec{w}; \vec{x}, y)
\end{equation*}
where $c_{+}, c_{-}$ (in the binary case) control the tradeoff.

In the case of perceptron and SVM, the constants change the
slope of the loss.
They also scale the gradient.

In the case of binary classification, specifying
$c_{+}$ and $c_{-}$ is equivalent to specifying
$\alpha = \frac{c_{+}}{c_{-}}$ and using
$c_{+}' = \alpha, c_{-} = 1$.

Instead of using a cost-sensitive classifier,
one can also use a single classifier and vary the
\emph{classification threshold} $\tau$
($y = sign(\vec{w}^T \vec{x} - \tau$).
This is similar to training multiple classifiers with
different cost-sensitive parameters, but has lower
computational cost.
However, optimising the cost-sensitive loss should lead to better results.


\subsection{Classification Metrics}
\begin{table}[h]
\begin{tabular}{lllll}
 & \multicolumn{3}{c}{True label}                                                                &      \\ \cline{3-4}
\multirow{3}{*}{Predicted label} & \multicolumn{1}{l|}{}         & \multicolumn{1}{l|}{Positive} & \multicolumn{1}{l|}{Negative} &      \\ \cline{2-4}
                                 & \multicolumn{1}{l|}{Positive} & \multicolumn{1}{l|}{TP}       & \multicolumn{1}{l|}{FP}       & $\Sigma = p_{+}$ \\ \cline{2-4}
                                 & \multicolumn{1}{l|}{Negative} & \multicolumn{1}{l|}{FN}       & \multicolumn{1}{l|}{TN}       & $\Sigma = p_{-}$ \\ \cline{2-4}
                                 &                               & $\Sigma = n_{+}$              & $\Sigma = n_{-}$ &     
\end{tabular}
\end{table}

A confusion matrix is a generalisation of the
table to $c$ classes where the columns are
the true class, the rows the predicted classes,
and the fields contain the counts.

Let $n$ be the total number of samples.
$n = p_{+} + p_{-} = n_{+} + n_{-}$.
Everything containing $n$ refers to the true data,
everything containing $p$ to the predictions.

General metrics are
\begin{description}
    \item[Accuracy]: \begin{equation*}
        \frac{TP + TN}{n}
    \end{equation*}
    \item[Precision]: \begin{equation*}
        \frac{TP}{TP + FP} = \frac{TP}{p_{+}}
    \end{equation*}
    \item[Recall / Sensitivity / TPR]: \begin{equation*}
        \frac{TP}{TP + FN} = \frac{TP}{n_{+}}
    \end{equation*}
    \item[Specificity / TNR]: \begin{equation*}
        \frac{TN}{FP + TN} = \frac{TN}{n_{-}}
    \end{equation*}
    \item[F1 Score]: \begin{equation*}
        \frac{2TP}{2TP + FP + FN} = \frac{2}{\frac{1}{prec} + \frac{1}{rec}} = 2 \cdot \frac{prec \cdot rec}{prec + rec}
    \end{equation*}
    \item[$F_\beta$ Score]: \begin{equation*}
        (1 + \beta^2) \cdot \frac{prec \cdot rec}{\beta^2 prec + rec}
    \end{equation*}
\end{description}
The F1 score is the
harmonic mean between precision and recall.
Precision and recall should not be averaged!
The $F_\beta$ score is a generalisation of the
F1 score.

TODO: Micro and macro averaging of F-scores.

TODO: Cross-entropy loss.

In the \emph{precision recall curve},
recall is used for the x-axis, precision for the y-axis.
Being in the top-right corner is better.

More metrics are
\begin{description}
    \item[True positive rate (TPR)]: \begin{equation*}
        \frac{TP}{TP + FN} = \frac{TP}{n_{+}} = recall
    \end{equation*}
    \item[False positive rate (FPR)]: \begin{equation*}
        \frac{FP}{TN + FP} = \frac{FP}{n_{-}}
    \end{equation*}
    \item[True negative rate (TNR)]: \begin{equation*}
        \frac{TN}{FP + TN} = \frac{TN}{n_{-}} = specificity
    \end{equation*}
\end{description}

Suppose class $+$ is predicted with probability $p$.
Then, $\mathbb{E}[TPR] = \mathbb{E}[FPR] = p$.
Thus, TPR and FPR establish a baseline:
If we predict labels at random with probability $p$,
the expected TPR and FPR are $p$.

The \emph{Receiver Operator Characteristic (ROC) Curve}
uses the FPR as the x-axis, TPR as the y-axis.
Being on the line with slope $1$ corresponds to random
guessing. Being closer to the bottom right corner is worse
than random guessing, being closer to the top left corner
is better.

\begin{theorem}
    Algorithm 1 dominates algorithm 2 in terms of the
    ROC Curve if and only if
    algorithm 1 dominates algorithm 2 in terms of the
    Precision Recall Curve.
    
    One algorithm dominates the other if its curve
    is strictly above the other curve.
\end{theorem}

The \emph{Area under the Curve (AUC)} describes the
area under the Precision Recall / ROC curve.
Thus, for ROC, $AUC = \frac{1}{2}$ corresponds to random
guessing, $AUC = 1$ is ideal.

\section{Multi-Class Classification}
In this section, let $y_i \in \mathcal{Y} = \{1, \dotsc, c\}$
and $D$ be the data set.
In a multi-class classification scenario,
we want a function $f : \mathcal{X} \to \mathcal{Y}$.

In one-vs-all (OVA) / one-vs-rest (OVR) classification,
$c$ binary classifiers $f^{(i)}$ are trained,
one for each class.
The positive examples are those from class $c$,
negative examples those from all other classes.

The prediction is then
\begin{equation*}
    \hat{y} = \arg\max_i f^{(i)}(\vec{x})
\end{equation*}
i.e. take the classifier with the highest confidence.

The confidence of a classification is given as
$|f^{(i)}(\vec{x})|$.
For any $\alpha > 0$, multiplying it to the function output
does not change the classification result, but the confidence.
One solution for a consistent confidence measure would thus be
to normalise the weights,
i.e. $\vec{w} \gets \frac{\vec{w}}{\norm{\vec{w}}_2}$.
However, in practice, if regularisation is used,
the magnitude $\norm{\vec{w}}_2$ is generally under control.

If unit magnitude for all weights is assumed,
we can consider the euclidian distance from a point
to each decision boundary.
Thus, in all overlapped areas, the decision boundaries
are separated by equal angles.

One-vs-all classification only works if classifiers produce
confidences of similar magnitude.
Furthermore, individual classifiers almost always see
a very imbalanced data set.
Also, some class might not be linearly separable from the
others, leading to failure.

In one-vs-one (OVO) classification,
$\frac{c (c - 1)}{2}$ classifiers are trained;
one for each distinct pair of classes $(i, j)$.
The positive samples are those from class $i$,
negative ones from class $j$.

The prediction is done using a voting scheme.
The class with the highest number of positive predictions wins.
However, ties may happen and need to be broken somehow.

The advantage of OVA is that it is faster as fewer classifiers
need to be trained. However, it requires confidence and leads
to class imbalance.
OVO does not have confidence or imbalance issues,
but is slower to train.

The alternatives are using other encodings or
multi-class models.

\section{Dimensionality Reduction}
The basic challenge is:
Given a data set
$D = \{\vec{x}_1, \dotsc, \vec{x}_n\} \in \mathbb{R}^d$,
find an \emph{embedding}
$\vec{z}_1, \dotsc, \vec{z}_n \in \mathbb{R}^k$
with $k < d$.

The embedding may then be used for visualisation,
regularisation (model selection),
unsupervised feature discovery,
and more.

Typically, a mapping $f : \mathbb{R}^d \to \mathbb{R}^k$
with $k \ll d$ is obtained.

In \emph{linear dimensionality reduction},
the mapping is of the form
$f(\vec{x}) = \vec{A}\vec{x}, \vec{A} \in \mathbb{R}^{k \times d}$.
In \emph{nonlinear dimensionality reduction},
other mappings (parametric or non-parametric) are used.

Linear dimensionality reduction can be interpreted as a form of
compression.
It should be possible to accurately reconstruct the original data
from the embedding.

\section{Probabilistic Modelling}
The goal of supervised learning is:
Given training data $D = \{(\vec{x}_1, y_1), \dotsc, (\vec{x}_n, y_n)\}$,
we want to identify a hypothesis $h : \mathcal{X} \to \mathcal{Y}$
and minimise the prediction error (risk).
The prediction error / true risk is
defined according to some loss function
$\ell$ and given as
\begin{equation*}
    R(h) = \int{P(\vec{x}, y) \ell(y; h(\vec{x})) d\vec{x} dy}
        = \mathbb{E}_{\vec{X}, Y}[\ell(Y; h(\vec{X})]
\end{equation*}

The fundamental assumption is that
the data set is generated i.i.d.,
i.e.
\begin{equation*}
    (\vec{x}_i, y_i) \overset{i.i.d.}{\sim} P(\vec{X}, Y)
\end{equation*}

In least squares regression,
the risk is
$R(h) = \mathbb{E}_{\vec{X}, Y}[(Y - h(\vec{X}))^2]$.
Assuming the true distribution $P(\vec{X}, Y)$ is known,
the $h$ minimising the risk is given as
\begin{align*}
    \min_h{R(h)}
    &= \min_h{\mathbb{E}_{\vec{X}, Y}[(Y - h(\vec{X}))^2]} \\
    &= \min_h{\mathbb{E}_{\vec{X}}[\mathbb{E}_Y[(Y - h(\vec{x}))^2 \mid \vec{X} = \vec{x}]]} \\
    &= \mathbb{E}_{\vec{X}}[
        \min_{h(\vec{x})}{\mathbb{E}_Y[(Y-h(\vec{x})^2 \mid \vec{X} = \vec{x}]}
    ]
\end{align*}
since $h$ can be considered independently for each $\vec{x}$.

Now, for a fixed $\vec{x}$, the optimal prediction is
\begin{equation*}
    y^*(\vec{x}) \in \arg\min_{\hat{y}}{
        \underbrace{\mathbb{E}_Y[(\hat{y} - y)^2]}_{= \ell(\hat{y})}
    }
\end{equation*}

This leads to
\begin{align*}
    \frac{d}{d \hat{y}} \ell(\hat{y}) &= \int{2 (\hat{y} - y) p(y \mid \vec{x}) dy} \overset{!}{=} 0 \\
    &\Rightarrow \underbrace{\int{\hat{y} \cdot p(y \mid \vec{x}) dy}}_{=\hat{y}} = \underbrace{\int{y \cdot p(y \mid \vec{x}) dy}}_{= \mathbb{E}[Y \mid \vec{X} = \vec{x}]} \\
    &\Leftrightarrow \hat{y} = \mathbb{E}[Y | \vec{X} = \vec{x}]
\end{align*}

Therefore, in least squares regression,
the hypothesis minimising the true risk
is given by the conditional mean
\begin{equation*}
    h^*(\vec{x}) = \mathbb{E}[Y \mid \vec{X} = \vec{x}]
\end{equation*}

This hypothesis is called the
\emph{Bayes' optimal predictor}.

In practice, with finite data, one strategy
is to estimate the conditional distribution
\begin{equation*}
    \hat{P}(Y \mid \vec{X})
\end{equation*}
and for a test point $\vec{x}$ to
predict the label
\begin{equation*}
    \hat{y} = \hat{\mathbb{E}}[Y \mid \vec{X} = \vec{x}]
    = \int{y \cdot \hat{P}(y \mid \vec{X} = \vec{x}) dy}
\end{equation*}


\subsection{Conditional Maximum Likelihood Estimation}
A common approach for conditional distribution estimation
is to choose a particular \emph{parametric form}
$\hat{P}(Y \mid \vec{X}, \theta)$
and optimise the parameters using
\emph{maximum conditional likelihood estimation}:
\begin{equation*}
    \theta^* = \arg\max_\theta{
        \hat{P}(y_1, \dotsc, y_n \mid \vec{x}_1, \dotsc, \vec{x}_n, \theta)
    }
\end{equation*}
By factoring the density (as the data is i.i.d.),
by applying the logarithm and switching the sign,
this is equivalent to minimising the
\emph{negative log likelihood}
\begin{equation*}
    \theta^* = \arg\min_\theta{
        -\sum_{i=1}^n{\log{\hat{P}(y_i \mid \vec{x}_i, \theta)}}
    }
\end{equation*}

We denote the negative log likelihood by $L(\theta)$.

MLE has several nice statistical properties:
\begin{description}
    \item[Consistency]: Parameter estimate
    converges to true parameters in probability.
    \item[Asymptotic efficiency]: Smallest
    variance among all well-behaved estimators
    for large $n$.
    \item[Asymptotic normality]
\end{description}


\subsubsection{Example: Regression with Gaussian Noise}
Assume $Y = h(\vec{X}) + \epsilon$,
$\epsilon \sim \mathcal{N}(0, \sigma^2)$ 
(with known $\sigma^2$) and
$h(\vec{x}) = \vec{w}^T \vec{x}$.
I.e., the model is linear and noise is Gaussian
with known variance.

Then,
$\hat{p}(y \mid \vec{X} = \vec{x}) = \mathcal{N}(y; \vec{w}^T \vec{x}, \sigma^2)$
and
\begin{equation*}
    \hat{\vec{w}} = \arg\min_{\vec{w}}{L(\vec{w})}
    = \arg\min_{\vec{w}}{
        -\sum_{i=1}^n{\log{\hat{p}(y_i \mid \vec{x}_i, \vec{w}, \sigma^2)}}
    }
\end{equation*}

Furthermore,
\begin{align*}
    - \log{\hat{p}(y_i \mid \vec{x}_i, \vec{w}, \sigma^2)}
    &= -\log{\mathcal{N}(y; \vec{w}^T \vec{x}, \sigma^2)} \\
    &= -\log{\frac{1}{\sqrt{2 \pi \sigma^2}} \exp{
        \left( -\frac{(y - \vec{w}^T \vec{x})^2}{2 \sigma^2} \right)
    }} \\
    &= \frac{1}{2} \log{(2 \pi \sigma^2)} + \frac{1}{2 \sigma^2} (y-\vec{w}^T \vec{x})^2
\end{align*}
and therefore
\begin{align*}
    \hat{\vec{w}} &= \arg\min_{\vec{w}}{
        \sum_{i=1}^n{\frac{1}{2} \log{(2 \pi \sigma^2)} + \frac{1}{2 \sigma^2} (y_i-\vec{w}^T \vec{x}_i)^2}
    } \\
    &= \arg\min_{\vec{w}}{
        \underbrace{\frac{n}{2} \log{(2 \pi \sigma^2)}}_{= \text{const w.r.t. $\vec{w}$}}
        + \frac{1}{2 \sigma^2} \sum_{i=1}^n{(y_i - \vec{w}^T \vec{x}_i)^2}
    } \\
    &= \arg\min_{\vec{w}}{\sum_{i=1}^n{(y_i - \vec{w}^T \vec{x}_i)^2}}
\end{align*}

Thus, under the conditional linear Gaussian assumption,
maximising the likelihood is equivalent
to least squares regression.


\subsubsection{MLE for i.i.d. Gaussian noise}
The above example holds more general.

Suppose $\mathcal{H} = \{h : \mathcal{X} \to \mathbb{R}\}$
is a class of functions.
Further assume
\begin{equation*}
    P(Y = y \mid \vec{X} = \vec{x}) =
    \mathcal{N}(y \mid h^*(\vec{x}), \sigma^2)
\end{equation*}
for some $h^* \in \mathcal{H}$ and some $\sigma^2 > 0$.

The MLE for data $D = \{(\vec{x}_1, y_1), \dotsc, (\vec{x}_n, y_n)\}$
is given by
\begin{equation*}
    \hat{h} = \arg\min_{h \in \mathcal{H}}{
        \sum_{i=1}^n{(y_i - h(\vec{x}_i))^2}
    }
\end{equation*}

Thus, the MLE is given by the least squares
solution, assuming noise is iid Gaussian
with constant variance.


\subsection{Bias-Variance Tradeoff}
While MLEs have nice statistical properties,
those only hold for $n \to \infty$.
For finite $n$, overfitting must be avoided.

The bias variance tradeoff states
for the sum of squared errors that
\begin{equation*}
    \text{Prediction error} = \text{Bias}^2 + \text{Variance} + \text{Noise}
\end{equation*}
where
\begin{description}
    \item[Bias] is the excess risk of the best
    \emph{considered} model,
    compared to minimal achievable risk
    knowing $P(\vec{X}, Y)$
    (i.e. given infinite data).
    Going away from the Bayes' optimal
    prediction, i.e. restricting function
    class, increases bias.
    \item[Variance] is the risk incurred
    due to estimating using finite data
    \item[Noise] is the risk incurred by
    optimal model (i.e. irreducible error)
\end{description}

Usually, bias and variance have to be traded
via model selection and regularisation.
High bias corresponds to underfitting
while high variance corresponds to overfitting.

The MLE solution depends on the training data $D$,
i.e. $\hat{h} = \hat{h}_D$.

We want to chose $\mathcal{H}$ to have small bias,
i.e. have small squared error on average:
\begin{equation*}
    \mathbb{E}_{\vec{X}}[
        \underbrace{\mathbb{E}_D[\hat{h}_D(\vec{X})] - h^*(\vec{X})}_{\text{Bias}}
    ]^2
\end{equation*}

The estimator itself is random and
thus has some variance:
\begin{equation*}
    \mathbb{E}_{\vec{X}}[\var_D[\hat{h}_D(\vec{X})]^2]
    = \mathbb{E}_{\vec{X}}[
        \mathbb{E}_D[(
            \hat{h}_D(\vec{X}) - \mathbb{E}_{D'}[\hat{h}_{D'}(\vec{X})]
        )^2]
    ]
\end{equation*}
TODO: No idea if the variance formula is correct...

Even if the Bayes' optimal hypothesis $h^*$ is
known, we would still incur some error du to noise:
\begin{equation*}
    \mathbb{E}_{\vec{X}, Y}[
        (Y - h^*(\vec{X}))^2
    ]
\end{equation*}

Ultimately, for least squares regression it holds that
\begin{align*}
    \underbrace{\mathbb{E}_D[\mathbb{E}_{\vec{X}, Y}[
        (Y - \hat{h}_D(\vec{X}))^2
    ]]}_{\text{Expected risk}}
\end{align*}
TODO: Week 18, slide 20

The MLE for linear regression is unbiased
if $h^* \in \mathcal{H}$.
Furthermore, it is the minimum variance
estimator among all unbiased estimators.
However, it may still overfit.
Thus, we can trade a small amount of bias
for a potentially large reduction in variance
by using regularisation.


\subsection{Maximum a Posteriori Estimate}
Bias can be introduced by expressing assumptions / a belief
on the parameters.
This is done using a \emph{Bayesian prior}.

Assume the parameters $\theta$ are from a known
\emph{prior distribution} which is fixed before any data
is observed.
Particularly, $\theta$ and the $\vec{X}_i$ are independent.

Then, the \emph{posterior distribution} of the parameter
$\theta$ is given using Bayes' rule
and $P(\theta \mid \vec{x}_{1:n}) = P(\theta)$ as
\begin{equation*}
    \underbrace{
        P(\theta \mid y_{1:n}, \vec{x}_{1:n})}_\text{posterior}
    = \frac{
        \overbrace{P(\theta)}^\text{prior}
        \overbrace{
        P(y_{1:n} \mid \theta, \vec{x}_{1:n})}^\text{likelihood}
    }{
        P(y_{1:n} \mid \vec{x}_{1:n})
    }
\end{equation*}

Finally, the \emph{maximum a posteriori (MAP) estimate}
is obtained by maximising $P(\theta \mid y_{1:n}, \vec{x}_{1:n})$
(or equivalently minimising the negative log term).
Note that $P(y_{1:n} \mid \vec{x}_{1:n})$ is constant w.r.t.
$\theta$ and can thus be ignored during optimisation.

MLE is a special case of MAP estimation.
Choosing a uniform parameter prior during
MAP estimation results in MLE.


\subsubsection{Example: Ridge Regression}
Assume $\theta = \vec{w} \sim \mathcal{N}(0, \beta^2 \vec{I})$,
i.e. the $w_i \sim \mathcal{N}(0, \beta^2)$ independently.

Then,
\begin{equation*}
    \arg\max_{\vec{w}}{
        P(\vec{w} \mid \vec{x}_{1:n}, y_{1:n})
    }
    = \arg\min_{\vec{w}}{
        -\log{P(\vec{w})}
        -\log{P(y_{1:n} \mid \vec{x}_{1:n}, \vec{w})}
        + \underbrace{\log{P(y_{1:n} \mid \vec{x}_{1:n})}}_\text{constant w.r.t. $\vec{w}$}
    }
\end{equation*}

For the negative prior log likelihood, we have
\begin{align*}
    -\log{P(\vec{w})} &=
        -\log{\prod_{i=1}^d{P(w_i)}} \\
    &= - \sum_{i=1}^d{\log{\mathcal{N}(w_i ; 0, \beta^2)}} \\
    &= - \sum_{i=1}^d{\log{
        \frac{1}{\sqrt{2 \pi \beta^2}}
        \exp{\left( -\frac{w_i^2}{2 \beta^2} \right)}
    }} \\
    &= \frac{d}{2} \log{2 \pi \beta^2}
    + \frac{1}{2 \beta^2} \sum_{i=1}^d{w_i^2} \\
    &= \text{const} + \frac{1}{2 \beta^2} \norm{\vec{w}}_2^2
\end{align*}

Combining the above form with the already known negative log
likelihood of a linear model with Gaussian noise, we get
\begin{align*}
    & \arg\min_{\vec{w}}{
        -\log{P(\vec{w})}
        -\log{P(y_{1:n} \mid \vec{x}_{1:n}, \vec{w})}
    } \\
    &= \arg\min_{\vec{w}}{
        \frac{1}{2\beta^2} \norm{\vec{w}}_2^2
        + \frac{1}{2\sigma^2} \sum_{i=1}^n{(y_i - \vec{w}^T \vec{x}_i)^2}
    } \\
    &= \arg\min_{\vec{w}}{
        \underbrace{\frac{\sigma^2}{\beta^2}}_\lambda \norm{\vec{w}}_2^2
        + \sum_{i=1}^n{(
        y_i - \vec{w}^T \vec{x}_i
        )^2}
    }
\end{align*}

Therefore, ridge regression is MAP estimation
for a linear regression problem, assuming that
the noise $P(y \mid \vec{x}, \vec{w})$ is
i.i.d. Gaussian and the prior $P(\vec{w})$
is Gaussian.

$\lambda$ is the ratio between the noise variance
and the prior variance.
Thus, a large $\lambda$ means that we believe to
have a lot of noise or the weights to be close to zero.


\subsubsection{Example: Lasso Regression}
Lasso regression / $L_1$-regularised linear regression
corresponds to a linear regression problem with Gaussian
noise and a Laplace parameter prior.

The density of a (univariate) Laplace distribution is given as
\begin{equation*}
    p(x; \mu, b) = \frac{1}{2b} \exp{\left(
        -\frac{|x - \mu|}{b}
    \right)}
\end{equation*}

The Laplace distribution has a spike around $\mu$.
With $\mu = 0$, this should bias the weights to be exactly zero.
However, the Laplace prior might not be optimal
for sparse solutions as we get non-zero weights
with probability 1 (it is continuous).

TODO: Slide 30.
The probability for a Gaussian distribution to have values far off from the mean is very low,
exponentially low.
The student-t distribution is heavier tailed, thus has only polynomial decay.
This can help substantially for robustness towards outliers.
Gaussian likelihood assumes that all points are very close to the prediction.
This is generally not true, especially if we have large outliers.
In those cases, we should use a different loss function,
i.e. replacing Gaussian with a more heavier-tailed distribution.

TODO: student's t-distribution slides 19, p. 9

\subsection{Regularisation via MAP inference}
Regularised estimation can often be understood as MAP inference.
The loss corresponds to the likelihood,
the regulariser to the parameter prior.
This leads to the form
\begin{equation*}
    \arg\min_{\vec{w}}{
        \sum_{i=1}^n{\ell(\vec{w}^T \vec{x}_i; \vec{x}_i, y_i)}
        + C(\vec{w})
    }
    = \arg\max_{\vec{w}}{
        \prod_{i=1}^n{P(y_i \mid \vec{x}_i, \vec{w})} P(\vec{w})
    }
    = \arg\max_{\vec{w}}{P(\vec{w} \mid D)}
\end{equation*}
where
\begin{align*}
    C(\vec{w}) &= - \log{P(\vec{w})} \\
    \ell(\vec{w}^T \vec{x}_i; \vec{x}_i, y_i)
        &= -\log{P(y_i \mid \vec{x}_i, \vec{w})}
\end{align*}

This allows to exchange priors (regularisers)
and likelihoods (loss functions).


\subsection{Classification}
In classification, the risk is
\begin{equation*}
    R(h) = \mathbb{E}_{\vec{X}, Y}[
        \underbrace{[Y \neq h(\vec{X})]}_\text{$1$ if $Y \neq H(\vec{X})$, $0$ otherwise}
    ]
\end{equation*}

Assuming $P(\vec{X}, Y)$ is known,
and $(\vec{x}_i, y_i) \sim P(\vec{X}, Y)$ iid,
we get
\begin{align*}
    h^*(\vec{x}) &= \arg\min_{\hat{y}}{
        \mathbb{E}_Y[[Y \neq \hat{y}] \mid \vec{X} = \vec{x}]
    } \\
    &= \arg\min_{\hat{y}}{
        \sum_{y=1}^c{P(Y = y \mid \vec{X} = \vec{x}) \cdot [y \neq \hat{y}] }
    } \\
    &= \arg\min_{\hat{y}}{
        \sum_{y : y \neq \hat{y}}{P(Y = y \mid \vec{X} = \vec{x})}    
    } \\
    &= \arg\min_{\hat{y}}{1 - P(Y = \hat{y} \mid \vec{X} = \vec{x})} \\
    &= \arg\max_{\hat{y}}{P(Y = \hat{y} \mid \vec{X} = \vec{x})}
\end{align*}

Thus, the \emph{Bayes' optimal predictor} (classifier)
is given by the most probable class:
\begin{equation*}
    h^*(\vec{x}) = \arg\max_y{P(Y = y \mid \vec{X} = \vec{x})}
\end{equation*}
A natural approach is therefore again to estimate $P(Y \mid \vec{X})$.


\subsection{MAP Learning Summary}
To summaries, learning through MAP inference
works as follows:
\begin{enumerate}
    \item Start with i.i.d. assumption
    on data points (can be relaxed).
    \item Choose likelihood function
    (results in loss).
    \item Choose prior distribution
    (results in regulariser).
    \item Optimise for MAP parameters.
    \item Choose hyperparameters
    via cross-validation.
    \item Make predictions via Bayesian
    decision theory.
\end{enumerate}

\section{Bayesian Decision Theory}
Let $P(y \mid \vec{x})$ be a conditional
distribution over labels,
$\mathcal{A}$ a \emph{set of actions},
and $C : \mathcal{Y} \times \mathcal{A} \to \mathbb{R}$
be a \emph{cost function}.

\emph{Bayesian decision theory} (BDT)
recommends to
pick the action which minimises the
\emph{expected cost}:
\begin{equation*}
    a^* =
    \arg\min_{a \in \mathcal{A}}{
        \mathbb{E}_y[C(y, a) \mid \vec{x}]
    }
\end{equation*}

If the true distribution $P(y \mid \vec{x})$
was known, this decision implements the
\emph{Bayes optimal decision}.
However, in practice, the distribution
can only be estimated.
If the action set is discrete,
calculations are often simplified
by calculating the expected cost
for each action individually
and then comparing them.


\subsection{Example: Logistic Regression}
In logistic regression,
the estimated conditional distribution is
\begin{equation*}
    \hat{P}(y \mid \vec{x}) =
    Ber(y ; \sigma(\hat{\vec{w}}^T \vec{x}))
\end{equation*}
the action set is
$\mathcal{A} = \{+1, -1\}$ and
the cost function
$C(y, a) = [y \neq a]$.

Then, the action which minimises the
expected cost is the most likely class:
\begin{align*}
    a^* &=
    \arg\min_{a \in \mathcal{A}}{
        \mathbb{E}_y[C(y, a) \mid \vec{x}]
    } \\
    &= \arg\max_y{\hat{P}(y \mid \vec{x})} \\
    &= sign(\vec{w}^T \vec{x})
\end{align*}


\subsection{Asymmetric Costs}
Logistic regression is used again,
however, with asymmetric cost.
The estimated conditional distribution is
\begin{equation*}
    \hat{P}(y \mid \vec{x}) =
    Ber(y ; \sigma(\hat{\vec{w}}^T \vec{x}))
\end{equation*}
the action set is
$\mathcal{A} = \{+1, -1\}$ and
the cost function is
\begin{equation*}
    C(y, a) =
    \begin{cases}
        c_{FP} & \text{if $y = -1$ and $a = +1$} \\
        c_{FN} & \text{if $y = 1$ and $a = -1$} \\
        0 & \text{otherwise}
    \end{cases}
\end{equation*}

Let $p = P(Y = +1 \mid \vec{x})$ be the
probability of a positive classification
for the input $\vec{x}$.
We can now defined the costs of positive
predictions as
\begin{equation*}
    c_+ = \mathbb{E}_y[C(y, +1) \mid \vec{x}]
    = (1 - p) \cdot c_{FP} + p \cdot 0
\end{equation*}
and for negative predictions as
\begin{equation*}
    c_- = \mathbb{E}_y[C(y, -1) \mid \vec{x}]
    = (1 - p) \cdot 0 + p \cdot c_{FN}
\end{equation*}

The optimal prediction for the estimated
distribution 
(using $p = P(y = +1 \mid \vec{x})$)
is then given as
\begin{align*}
    \text{predict} \hat{y} = +1
    &\Leftrightarrow c_+ < c_- \\
    &\Leftrightarrow (1-p) \cdot c_{FP} < p \cdot c_{FN} \\
    &\Leftrightarrow p > \frac{c_{FP}}{c_{FP} + c_{FN}}
\end{align*}


\subsection{Uncertainty}
Logistic regression is used again,
however, with an additional doubt action.
The estimated conditional distribution is
\begin{equation*}
    \hat{P}(y \mid \vec{x}) =
    Ber(y ; \sigma(\hat{\vec{w}}^T \vec{x}))
\end{equation*}
The action set is now
$\mathcal{A} = \{+1, -1, D\}$
where $D$ stands for \emph{doubt}.
For some doubt penalty $c$, the
new cost function is
\begin{equation*}
    C(y, a) =
    \begin{cases}
        [a \neq y] & \text{if $a \in \{+1, -1\}$} \\
        c & \text{if $a = D$}
    \end{cases}
\end{equation*}

Then, the action minimising the expected
cost is given by
\begin{equation*}
    a^* = \begin{cases}
        y & \text{if $\hat{P}(y \mid \vec{x}) \geq 1 - c$} \\
        D & \text{otherwise}
    \end{cases}
\end{equation*}


\subsection{Example: Least-Squares Regression}
In least-squares regression,
the estimated conditional distribution is
\begin{equation*}
    \hat{P}(y \mid \vec{x}) =
    \mathcal{N}(y ; \hat{\vec{w}}^T \vec{x}, \sigma^2)
\end{equation*}
the action set is
$\mathcal{A} = \mathbb{R}$ and
the cost function
$C(y, a) = (y - a)^2$.

Then, the action which minimises the
expected cost is the conditional mean:
\begin{align*}
    a^* &=
    \arg\min_{a \in \mathcal{A}}{
        \mathbb{E}_y[C(y, a) \mid \vec{x}]
    } \\
    &= \mathbb{E}[y \mid \vec{x}] \\
    &= \int{y \cdot \hat{P}(y \mid \vec{x}) dy} \\
    &= \hat{\vec{w}}^T \vec{x}
\end{align*}

If we chose an asymmetric cost
\begin{equation*}
    C(y, a) =
    \underbrace{c_1 \max{(y - a, 0)}}_\text{underestimation}
    +
    \underbrace{c_2 \max{(a - y, 0)}}_\text{overestimation}
\end{equation*}
then the action which minimises the
expected cost is
\begin{equation*}
    a^* = \hat{\vec{w}}^T \vec{x}
    + \sigma \cdot \Phi^{-1}\left(
    \frac{c_1}{c_1 + c_2}
    \right)
\end{equation*}
This shifts the prediction according
to the costs.


\subsection{Active Learning}
The goal of active learning is to minimise
the number of required labels.
The general idea is to pick examples
which the estimated conditional distribution
is most uncertain about.
This requires a uncertainty score $u_j$.
A simple one is
$u_j = -|\hat{P}(y_j = +1 \mid \vec{x}_j) - 0.5|$.
The most uncertain example is then
$j^* = \arg\max_j{u_j}$.

Given a pool of unlabelled examples
$D_X = \{\vec{x}_1, \dotsc, \vec{x}_n\}$
and an initially empty set of labelled
samples $D$.
For $t = 1, 2, \dotsc$, first
estimate $\hat{P}(Y \mid \vec{x})$
given the current data set $D$.
Then, pick the unlabelled example which
the current estimate is most uncertain
about:
\begin{equation*}
    i_t \in \arg\min_i{
        |0.5 - \hat{P}(Y_i \mid \vec{x}_i)|
    }
\end{equation*}
Finally, query the label $y_{i_t}$
and update the data set
$D \gets D \cup \{(\vec{x}_{i_t}, y_{i_t})\}$.

Active learning violates the i.i.d.
assumption!
It can thus get stuck with bad models.
There are also more advance selection
criteria,
e.g. querying the point which reduces
uncertainty of other points as much
as possible.

\section{Generative Modelling}
\emph{Discriminative models} aim to estimate
the conditional distribution
\begin{equation*}
    P(y \mid \vec{x})
\end{equation*}
\emph{Generative models} aim to estimate
the joint distribution
\begin{equation*}
    P(y, \vec{x})
\end{equation*}

Because discriminative models do not attempt
to model $P(\vec{x})$,
they are unable to detect outliers
and can thus become overconfident.

A conditional distribution can always be
derived from a joint one, but not vice versa:
\begin{equation*}
    P(y \mid \vec{x})
    = \frac{P(\vec{x}, y)}{P(\vec{x})}
    = \frac{P(\vec{x}, y)}{
        \int{P(\vec{x}, y) dy}
    }
\end{equation*}

The typical approach for generative
modelling is as follows:
\begin{enumerate}
    \item Estimate prior on labels: $P(y)$
    \item Estimate conditional distribution
    $P(\vec{x} \mid y)$ for each class $y$
    \item Obtain posterior (predictive) distribution
    using Bayes' rule:
    \begin{equation*}
        P(y \mid \vec{x}) =
        \frac{1}{Z} 
        \underbrace{P(y) P(\vec{x} \mid y)}_{P(\vec{x}, y)}
    \end{equation*}
\end{enumerate}
where $Z = P(\vec{x})$ is a normalisation
constant,
i.e. $Z = \sum_y{P(y) P(\vec{x} \mid y)}$.
If only $P(y \mid \vec{x})$ is desired,
the normalisation constant can be omitted.

Thus, generative modelling attempts to
infer the process according to which
examples are generated.

\subsection{Comparison}
In generative modelling, we have a
probabilistic model of each class.
The decision boundary is where one
model becomes more likely.
It allows natural usage of unlabelled data.
Discriminative modelling focuses
on the decision boundary.
It is more powerful (in terms of prediction)
if a lot of examples are available.
It also cannot use unlabelled data.

Generative models are always better if
the model is well-specified.
Otherwise, generative modelling is
better in case of small amount of data,
discriminative modelling in
case of large amount of data.


\subsection{Conjugate Distributions}
A pair of prior distribution and likelihood function
is called \emph{conjugate} if
the posterior distribution remains in
the same family as the prior.

Conjugate priors thus allow for regularisation
with almost no additional computational cost.

\subsubsection{List of Conjugate Distributions}
TODO: List of conjugate priors, slides 22 p. 36 and tutorial.

\subsubsection{Dirichlet Prior, Multinomial Likelihood}
The \emph{Dirichlet} prior is conjugate with
respect to the \emph{Multinomial} likelihood.

Let $\vec{X} = (X_1, \dotsc, X_n)$ and
$X_i \sim Categorical(\vec{\theta})$ iid.
$\vec{\theta} = (\theta_1, \dotsc, \theta_m)$
is the vector of the probabilities for
individual values.
Furthermore, let $N_j$ denote the
the number of times $j$ occurs in $\vec{X}$
and $\vec{N} = (N_1, \dotsc, N_m)$.

Then, $\vec{N} \sim Multi(\vec{\theta}, n)$
is from a \emph{multinomial} distribution.

The probability mass function is
\begin{equation*}
    P(\vec{N} \mid n, \vec{\theta}) =
    \underbrace{\frac{n!}{\prod_{j=1}^m{N_j!}}}_\text{normalisation constant}
    \prod_{j=1}^m{\theta_j^{N_j}}
\end{equation*}

For $Multi(\vec{\theta}, 1)$ is a categorical
distribution,
$Multi((\theta, 1 - \theta), n)$ equivalent to a binomial distribution.
The multinomial distribution is thus a
generalisation of the binomial distribution.

The \emph{Dirichlet distribution} has probability density function
\begin{equation*}
    P(\vec{\theta} \mid \vec{\alpha}) =
    \underbrace{\frac{\Gamma(\sum_{j=1}^m{\alpha_j})}{\prod_{j=1}^m{\Gamma(\alpha_j)}}}_\text{normalisation constant}
    \prod_{j=1}^m{\theta_j^{\alpha_j - 1}}
\end{equation*}
with $\alpha_j > 0$.
We write $\vec{\theta} \sim Dir(\vec{\alpha})$.

If we have a Dirichlet prior $P(\vec{\theta})$
and a multinomial likelihood $P(\vec{X} \mid \vec{\theta})$,
the resulting posterior $P(\vec{\theta} \mid \vec{X}) \propto P(\vec{X} \mid \vec{\theta}) P(\vec{\theta})$
is
\begin{equation*}
    P(\vec{\theta} \mid \vec{X}) = Dir(\vec{N} + \vec{\alpha})
\end{equation*}
Thus, $\vec{\alpha}$ act as pseudo-counts,
similar to the beta prior and binomial likelihood.

The Dirichlet prior is also conjugate with respect
to the Categorical likelihood.

The MLE of the multinomial likelihood is
\begin{equation*}
    \vec{\theta}_j^* = \frac{N_j}{n}
\end{equation*}
and the MAP estimate with a Dirichlet prior is
\begin{equation*}
    \vec{\theta}_j^* = \frac{N_j + \alpha_j - 1}{n + \sum_{j' = 1}^m{(\alpha_{j'} - 1)}}
\end{equation*}

