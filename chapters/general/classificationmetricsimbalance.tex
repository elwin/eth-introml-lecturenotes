\section{Classification Metrics and Class Imbalance}

\subsection{Class Imbalance}
Generally, the positive class is assumed to be rare.
There are multiple ways class imbalance can be mitigated:

\begin{description}
    \item[Subsampling] is removing majority class samples
    from the data set (e.g. uniformly at random).
    \item[Upsampling] is repeating data points from the
    minority class, possibly with small random perturbations.
    \item[Cost-sensitive classification] methods change
    the loss function to consider the class distribution.
\end{description}

Downsampling results in a smaller data set, thus training
becomes faster. However, data is wasted and information about
the majority class may be lost.
Upsampling makes use of all data but is slower and
perturbations requires arbitrary choices.
Cost-sensitive classification methods are a solution
to those problems.

In cost-sensitive classification,
the loss $\ell(\vec{w}; \vec{x}, y)$ is replaced as
\begin{equation*}
    \ell_{CS} = c_y \ell(\vec{w}; \vec{x}, y)
\end{equation*}
where $c_{+}, c_{-}$ (in the binary case) control the tradeoff.

In the case of perceptron and SVM, the constants change the
slope of the loss.
They also scale the gradient.

In the case of binary classification, specifying
$c_{+}$ and $c_{-}$ is equivalent to specifying
$\alpha = \frac{c_{+}}{c_{-}}$ and using
$c_{+}' = \alpha, c_{-} = 1$.

Instead of using a cost-sensitive classifier,
one can also use a single classifier and vary the
\emph{classification threshold} $\tau$
($y = sign(\vec{w}^T \vec{x} - \tau$)).
This is similar to training multiple classifiers with
different cost-sensitive parameters, but has lower
computational cost.
However, optimising the cost-sensitive loss should lead to better results.


\subsection{Classification Metrics}
\begin{table}[h]
\begin{tabular}{lllll}
 & \multicolumn{3}{c}{True label}                                                                &      \\ \cline{3-4}
\multirow{3}{*}{Predicted label} & \multicolumn{1}{l|}{}         & \multicolumn{1}{l|}{Positive} & \multicolumn{1}{l|}{Negative} &      \\ \cline{2-4}
                                 & \multicolumn{1}{l|}{Positive} & \multicolumn{1}{l|}{TP}       & \multicolumn{1}{l|}{FP}       & $\Sigma = p_{+}$ \\ \cline{2-4}
                                 & \multicolumn{1}{l|}{Negative} & \multicolumn{1}{l|}{FN}       & \multicolumn{1}{l|}{TN}       & $\Sigma = p_{-}$ \\ \cline{2-4}
                                 &                               & $\Sigma = n_{+}$              & $\Sigma = n_{-}$ &     
\end{tabular}
\end{table}

A confusion matrix is a generalisation of the
table to $c$ classes where the columns are
the true class, the rows the predicted classes,
and the fields contain the counts.

Let $n$ be the total number of samples.
$n = p_{+} + p_{-} = n_{+} + n_{-}$.
Everything containing $n$ refers to the true data,
everything containing $p$ to the predictions.

General metrics are
\begin{description}
    \item[Accuracy]: \begin{equation*}
        \frac{TP + TN}{n}
    \end{equation*}
    \item[Precision]: \begin{equation*}
        \frac{TP}{TP + FP} = \frac{TP}{p_{+}}
    \end{equation*}
    \item[Recall / Sensitivity / TPR]: \begin{equation*}
        \frac{TP}{TP + FN} = \frac{TP}{n_{+}}
    \end{equation*}
    \item[Specificity / TNR]: \begin{equation*}
        \frac{TN}{FP + TN} = \frac{TN}{n_{-}}
    \end{equation*}
    \item[F1 Score]: \begin{equation*}
        \frac{2TP}{2TP + FP + FN} = \frac{2}{\frac{1}{prec} + \frac{1}{rec}} = 2 \cdot \frac{prec \cdot rec}{prec + rec}
    \end{equation*}
    \item[$F_\beta$ Score]: \begin{equation*}
        (1 + \beta^2) \cdot \frac{prec \cdot rec}{\beta^2 prec + rec}
    \end{equation*}
\end{description}
The F1 score is the
harmonic mean between precision and recall.
Precision and recall should not be averaged!
The $F_\beta$ score is a generalisation of the
F1 score.

TODO: Micro and macro averaging of F-scores.

TODO: Cross-entropy loss.

In the \emph{precision recall curve},
recall is used for the x-axis, precision for the y-axis.
Being in the top-right corner is better.

More metrics are
\begin{description}
    \item[True positive rate (TPR)]: \begin{equation*}
        \frac{TP}{TP + FN} = \frac{TP}{n_{+}} = recall
    \end{equation*}
    \item[False positive rate (FPR)]: \begin{equation*}
        \frac{FP}{TN + FP} = \frac{FP}{n_{-}}
    \end{equation*}
    \item[True negative rate (TNR)]: \begin{equation*}
        \frac{TN}{FP + TN} = \frac{TN}{n_{-}} = specificity
    \end{equation*}
\end{description}

Suppose class $+$ is predicted with probability $p$.
Then, $\mathbb{E}[TPR] = \mathbb{E}[FPR] = p$.
Thus, TPR and FPR establish a baseline:
If we predict labels at random with probability $p$,
the expected TPR and FPR are $p$.

The \emph{Receiver Operator Characteristic (ROC) Curve}
uses the FPR as the x-axis, TPR as the y-axis.
Being on the line with slope $1$ corresponds to random
guessing. Being closer to the bottom right corner is worse
than random guessing, being closer to the top left corner
is better.

\begin{theorem}
    Algorithm 1 dominates algorithm 2 in terms of the
    ROC Curve if and only if
    algorithm 1 dominates algorithm 2 in terms of the
    Precision Recall Curve.
    
    One algorithm dominates the other if its curve
    is strictly above the other curve.
\end{theorem}

The \emph{Area under the Curve (AUC)} describes the
area under the Precision Recall / ROC curve.
Thus, for ROC, $AUC = \frac{1}{2}$ corresponds to random
guessing, $AUC = 1$ is ideal.
